\begin{thebibliography}{99}

\bibitem{Jeria} Jeria Carvajal, Esther.\\
Fenómeno Facebook.\\
Extraído el 01 de Mayo del 2011, de\\
http://www.bibliodigital.udec.cl/index.php?option=com\_content\& task=view\&id=113\&Itemid=9.

\bibitem{Rodriguez} Rodríguez Morales, Germania (2008, Mayo).\\
Educación Superior en Latinoamérica y la Web2.0.\\
Extraído el 24 de Abril del 2011, de\\
http://www.utpl.edu.ec/gcblog/wp-content/uploads/web2-y-educacion-superior.pdf.

\bibitem{Gonzalez} González Mariño, Julio Cesar (2006, Enero).\\
B-Learning utilizando software libre, una alternativa viable en Educación Superior.\\
Universidad Autónoma de Tamaulipas, México.\\
Extraído el 24 de Abril del 2011, de\\
http://revistas.ucm.es/edu/11302496/articulos/RCED0606120121A.PDF.

\bibitem{Bartolome} Bartolomé, Antonio (2004).\\
Blended Learning. Conceptos básicos.\\
Píxel-Bit. Revista de Medios y Educación, 23, pp. 7-20.\\
Universidad de Barcelona, España.\\
Extraído el 24 de Abril del 2011, de\\
http://www.lmi.ub.es/personal/bartolome/articuloshtml/04\_blend\-ed\_learning/documentacion/1\_bartolome.pdf.

\bibitem{Gilmore} Gilmore, Jason W. (2011).\\
Easy PHP Websites with the Zend Framework.\\

\end{thebibliography}
