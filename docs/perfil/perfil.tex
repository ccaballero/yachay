\documentclass[letterpaper,11pt]{article}
\usepackage[spanish]{babel}
\usepackage[utf8]{inputenc}

\usepackage{rotating}
\usepackage{multirow}

\usepackage{lmodern}
\usepackage[T1]{fontenc}
\usepackage{textcomp}

\usepackage[
pdfauthor={Carlos Eduardo Caballero Burgoa},%
pdftitle={Perfil de proyecto},%
colorlinks,%
citecolor=black,%
filecolor=black,%
linkcolor=black,%
urlcolor=black
pdftex]{hyperref}

\title{Perfil de proyecto}
\author{Carlos Eduardo Caballero Burgoa}

\begin{document}
\maketitle
\section{Introducción}
Con el auge de los últimos años con respecto a la red social Facebook \cite{Jeria}, se ha
notado un gran cambio en la mentalidad de las personas con respecto a su entorno,
compartiendo recursos e intercambiando ideas, se han abierto grandes posibilidades para un
salto en las viejas concepciones respecto a lo que concierne a las formas de aprendizaje y
la gestión del conocimiento.

Aunque los cambios han sido positivos, aún pueden concebirse nuevas e innovadoras
maneras para obtener una gran retroalimentación entre los estudiantes, con una forma
más de asistir a la educación en las aulas.

En este documento se plantea la construcción de una red social orientada a tópicos
netamente académicos, intentando de alguna manera reducir los métodos estrictamente
formales en la relación entre el educador y sus alumnos. Y de esta forma obtener una 
mayor integración entre estudiantes, docentes, fomentando de esa forma la interacción, 
comunicación y colaboración entre las partes.

\section{Antecedentes}
Con la creciente accesibilidad de las personas a internet, estas han ido cambiado su rol,
de simples consumidores de recursos, a participar en roles de creación, publicación,
categorización y valoración de los recursos, es decir “Pasar de ser consumidores de
información en Internet a ser productores de contenidos, información y conocimiento”.
\cite{Rodriguez}

Todo esto ha abierto un nuevo camino hacia nuevas formas de interrelación social, que
ofrecen una inmejorable oportunidad en el campo de lo educativo, colaborando en el
apoyo y mejora de los métodos de aprendizaje. Aprovechando los nuevos conceptos
acerca de Web2.0, que es una “Revolución social más que tecnológica, que da un énfasis
especial al intercambio abierto del conocimiento”. \cite{Rodriguez}

Redes sociales como Hi5, Facebook, MySpace, Orkut, LinkedIn entre otras, permiten a sus 
usuarios almacenar, organizar y compartir recursos como fotos, videos, etc. Además de
crear comunidades por entorno a intereses comunes de propósito general \cite{Rodriguez}.
También existen otras posibilidades mas orientadas a asistir al aprendizaje, como ser:
Moodle o Elgg, grandes sistemas que cuentan con el apoyo de muchas instituciones
educativas y desarrolladores, “que permiten al docente contextualizar al aula, la
utilización de las diferentes herramientas tecnológicas que tendrá a su disposición,
para atender las necesidades específicas de aprendizaje, que previamente haya
identificado en su labor docente”. \cite{Gonzalez}

\section{Definición del problema}
Se ha observado que los docentes se ven sobrecargados de actividades, que en parte
podrían ser simplificadas, ya sea en manejar toda la logística de un espacio virtual para su
materia, o en la misma atención que debe brindar a los estudiantes.

“El tutor debe atender a un elevado número de alumnos, ante la imposibilidad de atender
este trabajo se recurre a dejar de lado a aquellos alumnos que no insisten, se utilizan
mensajes genéricos o fragmentos de textos copiados y pegados sin excesivo cuidado, se
leen los mensajes de los alumnos de modo rápido, ignorando aspectos o matices
importantes.” \cite{Bartolome}

Además de notar que los estudiantes al ver el modelo actual que deben seguir en sus
estudios superiores, van perdiendo progresivamente el interés por compartir sus ideas y
experiencias; conocimiento que podría servir a otros estudiantes en la construcción de sus
propios criterios.

Pero en los estudiantes que ya poseen una solida rutina de participación la dificultad viene
sumida en la amplia variedad de sitios orientados a la provisión de recursos, despertando
una necesidad de centralizar todos estos recursos en un solo lugar.

Por lo mencionado se define el problema como:

\emph{“La escasa interacción académica entre los estudiantes conduce al uso de métodos
deficientes de adquisición del conocimiento.”}

\section{Objetivos}

\subsection{Objetivo General}
Promover el intercambio de información entre los estudiantes, mediante el uso de una red
social para mejorar los métodos de adquisición del conocimiento.

\subsection{Objetivos Específicos}
\begin{itemize}
\item Agilizar la creación de espacios virtuales para incrementar la cantidad y variabilidad de
estos.
\item Facilitar el intercambio de recursos entre los estudiantes para acelerar la adquisición de
experiencia.
\item Facilitar el intercambio de recursos entre distintas instancias de la red para mejorar la
disponibilidad de recursos.
\item Mejorar los canales de comunicación entre estudiantes y docentes para facilitar la
retroalimentación.
\item Planear estrategias que fomenten la participación para mantener activo el sistema.
\end{itemize}

\section{Ingeniería de proyecto}
Véase el cuadro \ref{ingenieriadeproyecto} en la página \pageref{ingenieriadeproyecto}.

\begin{sidewaystable}
\centering
\small
\begin{tabular}{|l|l|l|p{6.5cm}|l|}
\hline
Objetivo General & Causa & Objetivos Específicos & Actividades & Resultados \\
\hline
\multirow{15}{2.6cm}{Promover la interacción académica entre los estudiantes mediante el uso de
una red social para mejorar los métodos de adquisición del conocimiento.} &
\multirow{3}{3cm}{Pérdida de tiempo al crear espacios de interacción por parte de los docentes 
y estudiantes.} &
\multirow{3}{3.5cm}{Agilizar la creación de espacios virtuales para incrementar la cantidad y
variabilidad de estos.} &
Analizar las herramientas actuales para creación de espacios virtuales. &
\multirow{3}{2.5cm}{Modulo para la creación de espacios virtuales.} \\
\cline{4-4}
& & & Diseñar e implementar un método eficaz y cómodo para la manipulación de los espacios virtuales. & \\
\cline{4-4}
& & & Evaluar la usabilidad y amigabilidad de la herramienta desarrollada. & \\
\cline{2-5}
& \multirow{3}{3cm}{De\-sa\-pro\-ve\-cha\-mi\-en\-to del conocimiento y experiencia de los estudiantes.} &
\multirow{3}{3.5cm}{Facilitar el intercambio de recursos entre los estudiantes para acelerar la
adquisición de experiencia.} &
Diseñar e implementar de una arquitectura orientada a la provisión de recursos web. &
\multirow{3}{2.5cm}{Modulo para la publicación de recursos web multimedia.} \\
\cline{4-4}
& & & Diseñar e implementar de una arquitectura orientada al consumo de recursos web. & \\
\cline{4-4}
& & & Crear un API funcional para el acceso a la información. & \\
\cline{2-5}
& \multirow{3}{3cm}{Inadecuada conectividad entre las distintas plataformas web 2.0.} &
\multirow{3}{3.5cm}{Facilitar el intercambio de recursos entre distintas instancias de la red
para mejorar la disponibilidad de recursos.} &
Modelar un proceso de intercambio de información basado en servicios web en el sistema. &
\multirow{3}{2.5cm}{Modulo de conectividad entre instancias de la red social.} \\
\cline{4-4}
& & & Implementar estándares de servicios web en el sistema. & \\
\cline{4-4}
& & & Crear un API funcional para el intercambio de información. & \\
\cline{2-5}
& \multirow{3}{3cm}{Innecesaria sobrecarga de trabajo de los docentes en la atención a los estudiantes.} &
\multirow{3}{3.5cm}{Mejorar los canales de comunicación entre estudiantes y docentes para facilitar
la retroalimentación.} &
Crear espacios virtuales para las respectivas materias, grupos y dinámicas de grupo. &
\multirow{3}{2.5cm}{Módulos auxiliares para el apoyo de la enseñanza en aula.} \\
\cline{4-4}
& & & Diseñar e implementar métodos que mejoren los canales de comunicación. & \\
\cline{4-4}
& & & Crear indicadores para el seguimiento de la fluidez de comunicación en el sistema. & \\
\cline{2-5}
& \multirow{3}{3cm}{Progresiva pérdida del interés de parte de los estudiantes.} &
\multirow{3}{3.5cm}{Planear estrategias que fomenten la participación para mantener activo el sistema.} &
Revisar los métodos actuales de atracción de los usuarios en los sistemas multiusuario. &
\multirow{3}{2.5cm}{Módulos auxiliares de fomento a la participación de los usuarios.} \\
\cline{4-4}
& & & Diseñar e implementar los módulos definidos. & \\
\cline{4-4}
& & & Crear indicadores para el seguimiento del grado de interés en el sistema. & \\
\hline
\end{tabular}
\caption{Ingeniería de proyecto}
\label{ingenieriadeproyecto}
\end{sidewaystable}

\section{Justificación}
La construcción de una red social por definición está inmersa en ese mundo de vida propia
que es internet, por tanto se nutre de todo lo que ella puede proveer.

Se ve también un gran ahorro de tiempo, tanto para los estudiantes, que podrán reutilizar
material de otras personas, además de tenerlos a disposición en cualquier momento;
como para los docentes, que se verán apoyados en su misión de enseñanza por nuevos
canales de comunicación, facilitando así todo el proceso de enseñanza-aprendizaje.

En el aspecto social, promueve la comunicación y fomenta la comunión entre personas
con distintos grados de conocimiento, haciendo que unos puedan conocer y decidir que
caminos pueden seguir, y a otros mostrando las ventajas y/o desventajas que pueden
encontrar en el camino a sus objetivos.

\section{Innovación tecnológica}
Se plantea utilizar la tecnología provista por las librerías del framework Zend, para
desarrollar en el lenguaje de programación PHP, de modo que la herramienta pueda ser
fácilmente instalada en el común de los servidores de internet.

Así también se pretende utilizar toda la estructura provista por REST\footnote{Representational
State Transfer (REST): Técnica de arquitectura software para sistemas hipermedia distribuidos
como la World Wide Web.} para la implementación de servicios, tanto para provisión como
consumo de recursos web. Cabe resaltar también el uso de estándares comunes en la web 2.0, 
tanto OpenID\footnote{OpenID: Estándar de identificación digital descentralizado.},
OEmbed\footnote{OEmbed: Estándar para permitir insertar contenido multimedia en un sitio
consumidor desde otro sitio proveedor.} y OAuth\footnote{OAuth: Protocolo abierto que permite
autorización segura de un API de modo estándar y simple.}, para facilitar toda la mecánica de
interacción entre sitios web.

\section{Alcance}
El desarrollo de este sistema considera toda la interacción entre distintas instancias del
sistema, es decir, en lo que respecta a autentificación, consumo y provisión de recursos y
control de privilegios. Es necesario mencionar también que escapan de las funciones de
este sistema la interacción entre el sistema desarrollado y otras redes sociales, sea para
provisión o consumo de recursos.

Otra restricción impuesta será el registro cerrado para usuarios, esta será exclusivamente 
por vía de invitación, todo esto para crear una red social de conexiones lo menos dispersas
posibles.

\begin{thebibliography}{99}

\bibitem{Jeria} Jeria Carvajal, Esther.\\
Fenómeno Facebook.\\
Extraído el 01 de Mayo del 2011, de\\
http://www.bibliodigital.udec.cl/index.php?option=com\_content\& task=view\&id=113\&Itemid=9.

\bibitem{Rodriguez} Rodríguez Morales, Germania (2008, Mayo).\\
Educación Superior en Latinoamérica y la Web2.0.\\
Extraído el 24 de Abril del 2011, de\\
http://www.utpl.edu.ec/gcblog/wp-content/uploads/web2-y-educacion-superior.pdf.

\bibitem{Gonzalez} González Mariño, Julio Cesar (2006, Enero).\\
B-Learning utilizando software libre, una alternativa viable en Educación Superior.\\
Universidad Autónoma de Tamaulipas, México.\\
Extraído el 24 de Abril del 2011, de\\
http://revistas.ucm.es/edu/11302496/articulos/RCED0606120121A.PDF.

\bibitem{Bartolome} Bartolomé, Antonio (2004).\\
Blended Learning. Conceptos básicos.\\
Píxel-Bit. Revista de Medios y Educación, 23, pp. 7-20.\\
Universidad de Barcelona, España.\\
Extraído el 24 de Abril del 2011, de\\
http://www.lmi.ub.es/personal/bartolome/articuloshtml/04\_blend\-ed\_learning/documentacion/1\_bartolome.pdf.

\end{thebibliography}

\end{document}
