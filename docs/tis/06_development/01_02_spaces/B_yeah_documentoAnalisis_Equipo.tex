\documentclass[letterpaper,spanish]{article}
\usepackage{babel}
\usepackage[latin1]{inputenc}
\usepackage[pdftex]{graphicx}
\usepackage[dvipsnames,usenames]{color}
\usepackage{fancyhdr}
\usepackage{latexsym}
\usepackage{verbatim}

\oddsidemargin -0.3cm \topmargin -1.1cm \headheight 2.5cm
\textwidth 16.8cm \textheight 19.8cm

\newsavebox{\fondo}
\sbox{\fondo}{\includegraphics[keepaspectratio,height=1.35\textheight,width=1.35\textwidth]{../../02_publicimage/margin.png}}

\pagestyle{fancy}
\fancyhead[C]{\setlength{\unitlength}{1in}
\begin{picture}(0,0)\put(-3.9,-9.3){\usebox{\fondo}}\end{picture}}

\renewcommand{\headrulewidth}{0pt}
\title{\huge{\textbf{DOCUMENTO DE AN\'ALISIS \\ Gesti\'on de equipos}}}
\author{Yeah! S.R.L.}
\date{\small{2 de abril 2009}}

\begin{document}
\maketitle \pagebreak \tableofcontents \pagebreak

\section{INTRODUCCI\'ON}

Para el buen funcionamiento del sistema de gesti\'on de cursos y notas, es necesario tener equipos de 
trababajo en algunos grupos de las diferentes materias, donde el profesor puede crear los equipos de 
distinta manera. 

\section{DESCRIPCI\'ON DEL PROBLEMA}

Como ya se menciono en la introducci\'on es necesario la creaci\'on y la evaluaci\'on de los equipos de los 
estudiantes.

Lo que la empresa TIS requiere para la creaci\'on de equipos es: \\
\begin{itemize}
	\item Creaci\'on de equipos seg\'un al orden de lista (apellido paterno,nombres).
        \item Creaci\'on de equipos aleatoriamente(nadie sabe quien ser\'a su compa\~nero).
        \item Asignaci\'on Manual.

\end{itemize} 

\section{DESCRIPCI\'ON DE LA SOLUCI\'ON}

La creaci\'on de equipos ser\'a un modulo que el profesor de cada materia podr\'a habilitar cuando lo 
requiera.

\subsection{Creaci\'on de equipos seg\'un al orden de lista (apellido paterno,nombres)}
\begin{itemize}
    \item El docente habilita  el soporte de equipos en el grupo.
    \item El docente pulsa crear equipos por orden de lista de $n$ estudiantes.
\end{itemize}

\subsection{Creaci\'on de equipos aleatoriamente(nadie sabe quien ser\'a su compa\~nero)}
\begin{itemize}
    \item El docente habilita el soporte de equipos en el grupo.
    \item El docente pulsa crear equipos aleatoriamente de $n$ estudiantes.
\end{itemize}

\subsection{Asignaci\'on Manual}
\begin{itemize}
    \item El docente arma los equipos a su criterio.
\end{itemize}

\section{MODULOS PROPUESTOS}

\begin{description}
    \item [SUBJECT] Se encarga de manejar las materias su descripci\'on de estas y ademas se encarga de editar, 
    crear o y eliminar grupos.
    \item [GROUP] Se encarga del control de su grupo de usuarios(o estudiantes).
    \item [USER] El modulo USER es el que se encarga de organizar a los diferentes tipos de usuarios
    del sistema mediante sus roles.
\end{description}

\section{FUNCIONES PROPUESTAS}
    \subsection{SUBJECT}
        \begin{description}
            \item [Ver materias] Lista todas las materias con un resumen de sus grupos y la descripci\'on de la 
            materia.
            \item [Editar materia] Permite editar el t\'itulo y la descripci\'on de la materia, y ademas de poder 
            eliminar a los grupos de la misma.
            \item [Ver grupos] Lista todos los grupos, con las opciones de editar a los mismos.
        \end{description}
    
    \subsection{GROUP}
        \begin{description}
            \item [Ver grupo] Muestra el detalle del grupo, y permite su edici\'on del mismo.
            \item [Ver usuarios] Lista a todos los usuarios(o estudiantes) de ese grupo.
            \item [Enviar invitaci\'on] Permite invitar a personas a unirse al grupo.
        \end{description}
    \subsection{USER}
        \begin {description}
            \item [Ver datos personales] Permite ver al usuario sus datos.
            \item [Editar datos personales] Permite al usuario editar sus datos.
        \end {description}

\section{RESTRICCIONES ENCONTRADAS}
Por el corto espacio de tiempo solo se desarrollaran tres formas de creaci\'on de equipos, quedando cuatro como 
ideas planteadas para desarrollar en un futuro.

\subsection {Creaci\'on de equipos seg\'un los estudiantes}
\begin{itemize}
    \item El docente habilita el soporte de equipos en el grupo con un numero m\'aximo y un m\'inimo.
    \item Se habilita la creaci\'on de los grupos para los estudiantes.
    \item Un estudiante puede crear un equipo.
    \item Un estudiante puede eliminar el equipo si esta solo.
    \item El estudiante puede inscribirse o entrar al equipo X creado por algun estudiante.
    \item Los estudiantes pueden abandonar el equipo o cambiar de equipo seg\'un el tiempo limite asignado 
    por el docente.
    \item El docente verif\'ica los grupos completos en la fecha limite.
    \item Si existen equipos incompletos el docente puede dar mas plazo para que se unan, puede asignar a 
    otros grupos aumentando su tama\~no, puede permitir que el equipo este conformado como esta, puede hacer 
    una fusi\'on de equipos peque\~nos.
\end{itemize}

\subsection{Balanceo por puntuaci\'on}
\begin{itemize}
    \item El sistema arma autom\'aticamente los grupos balanceando la puntuaci\'on de los estudiantes.
\end{itemize}

\subsection{Afinidad de intereses}
\begin{itemize}
    \item El sistema verif\'ica que los estudiantes tengan completo su perfil mas sus intereses.
    \item El sistema arma los equipos mediante el perfil de cada usuario.
\end{itemize}

\subsection{Por seleccion del alumno}
\begin{itemize}
    \item El docente designa capitanes para cada equipo.
    \item Los capitanes designados arman sus equipos escogiendo entre todos los alumnos de la clase.
\end{itemize}

\begin{comment}
\vspace{8cm}
    \begin{table}[htbp]
        \begin{center}
            \begin{tabular}{c c}
\_\_\_\_\_\_\_\_\_\_\_\_\_\_\_\_\_\_\_\_\_\_\_\_\_\_\_\_\_\_\_\_\_\_\_\_\_\_\_\_\_\_\_ &
\_\_\_\_\_\_\_\_\_\_\_\_\_\_\_\_\_\_\_\_\_\_\_\_\_\_\_\_\_\_\_\_\_\_\_\_\_\_\_\_\_\_\_ \\
Lic. Leticia Blanco Coca & Luis Arce Claros \\
Representante legal de la empresa TIS & Director del proceso de analisis de la empresa Yeah! S.R.L.\\
            \end{tabular}
        \label{tab1}
        \end{center}
    \end{table}
\end{comment}

\end{document}
